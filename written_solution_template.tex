\documentclass{article}
\usepackage[utf8]{inputenc}
\usepackage{fancyhdr}
\usepackage{lastpage}
\usepackage{amsfonts}
\usepackage{amsmath}
\usepackage{amssymb}
\usepackage{bm}
\usepackage{bbm}

\usepackage[shortlabels]{enumitem}
\usepackage[noabbrev, capitalise]{cleveref}

\usepackage{geometry}
 \geometry{
 a4paper,
 top=20mm,
 bottom=25mm,
 left=25mm,
 right=25mm,
 }

% define your IDs here:
\newcommand{\firststudentid}{123456789}
\newcommand{\secondstudentid}{987654321}

\pagestyle{fancy}
\fancyhf{}
\rhead{Written Solution for Assignment 1}
\chead{\firststudentid \qquad \secondstudentid}
\lhead{Natural Language Processing}
\rfoot{Page \thepage \hspace{1pt} of \pageref{LastPage}}
\renewcommand{\footrulewidth}{1pt}
 
\setlength{\parindent}{0pt}
\setlength{\parskip}{1em}
\renewcommand{\baselinestretch}{1.25}

\renewcommand{\thesubsection}{\thesection.\alph{subsection}}
\renewcommand{\thesubsubsection}{\thesubsection.\roman{subsubsection}}

\begin{document}

\refstepcounter{section}
\section{Gradient of Softmax + CE}
\subparagraph{•}
Denote SCE - Softmax Cross Entropy function:\\
$SCE(\theta, y) = CE(\hat{y}, y) = CE(softmax(\theta), y) \\$
$= -\sum_{i} y_i log(\hat{y}_i)\\$
$if \ k \ - \ true \ label:\\$
$= - log(\hat{y}_i) = -log(softmax(\theta)_k) = -log(\frac{exp(\theta_k)}{\sum_{j}exp(\theta_j)})\\$
$= log(\sum_{j}exp(\theta_j))-\theta_k\\$
$\Longrightarrow SCE(\theta, y) = log(\sum_{j}exp(\theta_j))-\theta_k\\$

\subparagraph{•}
Now let's calculate it's derivative with respect to some $\theta_i:\\$
$\frac{\delta SCE}{\delta \theta_i} = \frac{log(\sum_{j}exp(\theta_j))}{\delta \theta_i} -\frac{\delta \theta_k}{\delta \theta_i}\\$
$\downarrow \\$
$denote: \ f \ = \sum_{j}exp(\theta_j)\\ $
$\downarrow \\$
$\frac{log(\sum_{j}exp(\theta_j))}{\delta \theta_i} -\frac{\delta \theta_k}{\delta \theta_i}\\$
$= \frac{log(f)}{\delta f}\frac{\delta f}{\delta \theta_i} -\frac{\delta \theta_k}{\delta \theta_i}\\$
$= \frac{1}{f} \sum_{j} \frac{exp(\theta_j)}{exp(\theta_i)} -\frac{\delta \theta_k}{\delta \theta_i}\\$
$= \frac{1}{f} exp(\theta_i) -\frac{\delta \theta_k}{\delta \theta_i}\\$
$= \frac{exp(\theta_i)}{\sum_{j} exp(\theta_j)}-\frac{\delta \theta_k}{\delta \theta_i}\\$
$= \frac{exp(\theta_i)}{\sum_{j} exp(\theta_j)} - \mathbbm{1}(i=k)\\$
$= softmax(\theta)_i - \mathbbm{1}(i=k)\\$
$\Longrightarrow \nabla_\theta SCE = softmax(\theta) - y = \hat{y}-y\\$ 

\subparagraph{•}
Basically the gradient of this function is prediction softmax vector minus true label one-hot vector.

\section{Gradients of NN}
\subparagraph{•}
In this section we use:\\
$h=\sigma(x W_1 + b_1) \\$
$\theta = h W_2 + b_2 \\$
$\hat{y}= softmax(\theta) \\$
$CE(y,\hat{y}) = -\sum_{i} y_i log(\hat{y}_i)\\$
$\nabla_\theta SCE = \hat{y}-y$

\subparagraph{•}
Also notice that Jacobian matrix of a vector function sigmoid is:\\
$da(\sigma (a))=diag(\sigma' (a))= \begin{bmatrix}
\sigma' (a_1) & 0 & 0 & 0\\
0 & \sigma' (a_2) & 0 & 0\\
0 & 0 & \sigma' (a_i) & 0\\
0 & 0 & 0 & \sigma' (a_n) 
\end{bmatrix} \ ; \ \sigma' (a_i)=\sigma(a_i)(1-\sigma(a_i))$

\subparagraph{•}
Let's write a formula for each gradient using differentials:

\subparagraph{•}
$dx(SCE)=dx(SCE(hW_2+b_2)) \\$
$= \nabla_\theta SCE \ dx(hW_2+b_2) \\$
$= \nabla_\theta SCE \ dx(hW_2) \\$
$= \nabla_\theta SCE \ W_2^T \ dx(h) \\$
$= \nabla_\theta SCE \ W_2^T \ dx(\sigma (xW_1+b_1)) \\$
$= \nabla_\theta SCE \ W_2^T \ diag(\sigma' (xW_1+b_1)) \ dx(xW_1+b_1) \\$
$= \nabla_\theta SCE \ W_2^T \ diag(\sigma' (xW_1+b_1)) \ W_1^T \ dx \\$
$\Longrightarrow \nabla_x SCE = \nabla_\theta SCE \ W_2^T \ diag(\sigma' (xW_1+b_1)) \ W_1^T \\$
$\Longrightarrow \nabla_x SCE = (\hat{y}-y) \ W_2^T \ diag(\sigma' (xW_1+b_1)) \ W_1^T$

This formula won't work for batched data, but we can write it in the more generic way using Hadamard product, i.e. elementwise multiplication - $\circ$ :

$\Longrightarrow \nabla_x SCE = \{ (\hat{y}-y) \ W_2^T \ \circ\ \sigma' (xW_1+b_1)\} \ W_1^T$

\subparagraph{•}
Let's sanity check it's dimentions:\\
$x - 2 \times 10 \ $ (batch of two vectors)\\
$W_1 - 10 \times 7 \\$
$b_1 - 1 \times 7 \\$
$\sigma(xW_1+b_1) - 2 \times 7 $ (bias is broadcasted)\\
$\sigma'(xW_1+b_1) - 2 \times 7 $ (bias is broadcasted)\\
$W_2 - 7 \times 5 \\$
$b_2 - 1 \times 5 \\$
$y - 2 \times 5 $ (batch of two vectors)\\

$\Longrightarrow \nabla_x SCE = \{ (\hat{y}-y) \ W_2^T \ \circ\ \sigma' (xW_1+b_1)\} \ W_1^T\\$
$\Longrightarrow \ 2 \ \times \ 10 \  = \{ (2 \times 5) \ (5 \times 7) \ \circ\ (2 \times 7)\} \ (7 \times 10)$

All dimensions match.

\subparagraph{•}
$dW_1(SCE)=dW_1(SCE(hW_2+b_2)) \\$
$= \nabla_\theta SCE \ dW_1(hW_2+b_2) \\$
$= \nabla_\theta SCE \ dW_1(hW_2) \\$
$= \nabla_\theta SCE \ W_2^T \ dW_1(h) \\$
$= \nabla_\theta SCE \ W_2^T \ dW_1(\sigma (xW_1+b_1)) \\$
$= \nabla_\theta SCE \ W_2^T \ diag(\sigma' (xW_1+b_1)) \ dW_1(xW_1+b_1) \\$
$= x^T \ \nabla_\theta SCE \ W_2^T \ diag(\sigma' (xW_1+b_1)) \ dW_1 \\$
$\Longrightarrow \nabla_{W_1} SCE = x^T \ \nabla_\theta SCE \ W_2^T \ diag(\sigma' (xW_1+b_1)) \\$
$\Longrightarrow \nabla_{W_1} SCE = x^T \ (\hat{y}-y) \ W_2^T \ diag(\sigma' (xW_1+b_1))$

Same trick with Hadamard product, i.e. elementwise multiplication - $\circ$ :\\
$\Longrightarrow \nabla_{W_1} SCE = x^T \ \{ (\hat{y}-y) \ W_2^T \ \circ \ \sigma' (xW_1+b_1) \}$

\subparagraph{•}
$db_1(SCE)=db_1(SCE(hW_2+b_2)) \\$
$= \nabla_\theta SCE \ db_1(hW_2+b_2) \\$
$= \nabla_\theta SCE \ db_1(hW_2) \\$
$= \nabla_\theta SCE \ W_2^T \ db_1(h) \\$
$= \nabla_\theta SCE \ W_2^T \ db_1(\sigma (xW_1+b_1)) \\$
$= \nabla_\theta SCE \ W_2^T \ diag(\sigma' (xW_1+b_1)) \ db_1(xW_1+b_1) \\$
$= \nabla_\theta SCE \ W_2^T \ diag(\sigma' (xW_1+b_1)) \ 1 \ db_1 \\$
$\Longrightarrow \nabla_{b_1} SCE = \nabla_\theta SCE \ W_2^T \ diag(\sigma' (xW_1+b_1)) \\$
$\Longrightarrow \nabla_{b_1} SCE = (\hat{y}-y) \ W_2^T \ diag(\sigma' (xW_1+b_1))$

Same trick with Hadamard product, i.e. elementwise multiplication - $\circ$ :\\
$\Longrightarrow \nabla_{b_1} SCE \approx \{ (\hat{y}-y) \ W_2^T \} \ \circ \ \sigma' (xW_1+b_1)$

Notice that dims don't match, because bias was broadcasted. To match the dims, we sum the gradient along batch dimension.

$\Longrightarrow \nabla_{b_1} SCE = \sum_{batch=0}^N \{ \{ (\hat{y}-y) \ W_2^T \} \ \circ \ \sigma' (xW_1+b_1)\} _{batch}$

\subparagraph{•}
$dW_2(SCE)=dW_2(SCE(hW_2+b_2)) \\$
$= \nabla_\theta SCE \ dW_2(hW_2+b_2) \\$
$= \nabla_\theta SCE \ dW_2(hW_2) \\$
$= \nabla_\theta SCE \ h \ dW_2 \\$
$= \sigma (xW_1+b_1)^T \ \nabla_\theta SCE \ dW_2 \\$
$\Longrightarrow \nabla_{W_2} SCE = \sigma (xW_1+b_1)^T \ \nabla_\theta SCE \\$
$\Longrightarrow \nabla_{W_2} SCE = \sigma (xW_1+b_1)^T \ (\hat{y}-y)$

\subparagraph{•}
$db_2(SCE)=db_2(SCE(hW_2+b_2)) \\$
$= \nabla_\theta SCE \ db_2(hW_2+b_2) \\$
$= \nabla_\theta SCE \ 1 \ dW_2 \\$
$\Longrightarrow \nabla_{b_2} SCE = \nabla_\theta SCE \\$
$\Longrightarrow \nabla_{b_2} SCE \approx (\hat{y}-y) $

Adjust dims:\\
$\Longrightarrow \nabla_{b_2} SCE = \sum_{batch=0}^N (\hat{y}-y)_{batch} $

\section{GRU question}
\paragraph{Advantage}
Smaller discrete space - There are about 100 English-language characters in common usage if we include all punctuation marks. By contrast, a vocabulary is many thousands of words. For char-based model we need about 100 embeddings to represent all possible tokens.

\paragraph{Disadvantage}
Char-level models can generate unusual words. Word-level models can't generate mistyped words as these are not in their vocabulary. 



\end{document}